\chapter{Descripción del código maestro \textbf{Coupling}}
\label{C:coupling}
%\chapterquote{Negociemos Don Inodoro}{Fernando de la R\'{u}a, 2001}

\textbf{Coupling} es un código escrito en \textit{Fortran 90}, desarrollado por Jorge Leiva, Pablo Blanco y Gustavo Buscaglia.
Fue diseñado como programa \textit{maestro} en las funciones de acoplamiento explícito e implícito,
estableciendo la comunicación con diversos programas \textit{esclavos} mediante funciones de MPI.
Su desempeño fue validado en diversos trabajos de acoplamiento \cite{coup-0d3d} \cite{coup-black} \cite{coup-hyd} \cite{coup-strong}.

\section{Modelado de problemas acoplados}
\label{ap1:definicion}
El programa \textbf{Coupling} resuelve problemas que fueron formulados mediante el Método de Descomposición Disjunta de Dominios descripto en el \hyperlink{chapter.1}{Capítulo 1}.
El esquema de resolución fue diseñado de tal forma que se permite que las interfaces de acoplamiento cuenten con múltiples pares de incógnitas vectoriales en diferentes nodos.
Las principales variables en la definición de un problema son las siguientes:
\begin{itemize}
\item Cantidad de subdominios acoplados (variable \textit{Ncomp}).
\item Cantidad de uniones globales (variable \textit{Nbond}).
\item Cantidad de superficies en cada subdominio (variable \textit{Nsupc}).
\item Matriz de conectividades entre las interfaces de diferentes subdominios (variable con numeraciones locales \textit{Localsur} y globales \textit{Incid}.
\item Cantidad de pares de variables incógntias en cada interfaz (variable \textit{Npair}).
Se supone que cada incógnita de cada interfaz está acompañada de otra variable incógnita relacionada con su gradiente.
Por ejemplo, \textit{temperatura} y \textit{flujo de calor}, o \textit{velocidad} y \textit{presión}.
\item Cantidad de nodos con incógnitas en cada interfaz (variable \textit{Nodpair}).
\item Cantidad de componentes escalares en cada nodo (variable \textit{Npairc}).
\item Tipo de condición de borde para cada nodo (variable \textit{indexbc}).
Las opciones posibles son condiciones de borde \textit{esenciales} (tipo \textit{Dirichlet}) o \textit{naturales} (tipo \textit{Neumann}), y deben establecerse por separado para cada subdominio.
\end{itemize}
Para que el problema de acoplamiento quede bien definido,
todos estos datos deben ser provistos en el archivo de configuración \texttt{mesh.cfg} utilizando palabras claves específicas.
Además, debe proveerse un archivo de configuración extra, el arhivo \texttt{couple.cfg}.
En este archivo se detalla el método de acoplamiento a utilizar en la resolución del problema,
y sus diferentes parámetros específicos.
También se detallan los parámetros necesarios en problemas de evolución.

En la sección \ref{ap1:ejemplo} se describe como ejemplo la formulación utilizada en una de las aplicaciones analizadas en el \hyperlink{chapter.3}{Capítulo 3}.

\section{Metodología de resolución}
\label{ap1:resolucion}

El archivo \texttt{src/couple\_ddm.for} contiene la estructura principal del programa.
El modelo de comunicación entre programas implementado es el de paso de mensajes entre programas ejecutados en modos SISD o SIMD.
Conforme a la estrategia de acoplamiento descripta en el \hyperlink{chapter.2}{Capítulo 2}, se siguen los siguientes pasos:
\begin{enumerate}
\item Lectura de archivos de datos \texttt{mesh.cfg} y \texttt{couple.cfg}.
\item Inicialización de las funciones de MPI y publicación de puertos de conexión.
\item Establecimiento de la comunicación con los diferentes programas esclavos.
\item Envío de datos generales del problema para chequeo interno en cada programa.
\item Ingreso en el \textit{loop} de evolución.
\begin{enumerate}
\item Envío de valores \textit{guess} para las condiciones de borde de cada subdominio a cada programa \textit{esclavo}.
\item Recepción de valores calculados para las incógnitas de cada subdominio de cada programa \textit{esclavo}.
\item Resolución de las ecuaciones de residuos \ref{ecuaciones-residuos}.
\item Evaluación de la convergencia:
\begin{itemize}
\item si los resultados están convergidos el problema puede continuar con el siguiente paso de evolución;
\item si los resultados no están convergidos, es necesario reiniciar el cálculo.
\end{itemize}
\item Envío de orden de avance o de reinicio a cada programa \textit{esclavo}.
\end{enumerate}
\item Finalización de \textit{loop} de evolución.
\item Cierre de las conexiones con los programas \textit{esclavos} y finalización.
\end{enumerate}
Los pasos descriptos requieren una arquitectura de acoplamiento específica montada en los códigos \textit{esclavos}, la cual fue comentada en la sección \ref{2:arquitectura-mpi}.

\section{Ejemplo de uso}
\label{ap1:ejemplo}

A continuación se comenta la formulación del problema analizado en la sección \ref{3:mock-up}:
\textit{Análisis del segundo sistema de parada de un reactor de investigación}.
Se definen dos subdominios: el primero es el subsistema tri-dimensional arreglo de válvulas y el segundo es el subsistema cero-dimensional del tanque del reactor y porción superior de la red hidráulica.
Existe una única unión entre ambos subdominios, la \textit{Unión 1}.
Es necesario definir una numeración global de interfaces de acoplamiento.
El subsistema tri-dimensional contiene la interfaz global \textit{Interfaz 1} y el subsistema cero-dimensional contiene la interfaz global \textit{Interfaz 2}.
En cada interfaz existe un solo par de variables acopladas, el par \textit{caudal-presión}.
El \textit{caudal} es impuesto como condición de borde \textit{esencial} al \textit{Subdominio 1} y la \textit{presión} como condición de borde \textit{natural} al \textit{Subdominio 2}.
La Figura \ref{ra10-coupling} resume las definiciones previas.

\begin{figure}[h]
\centering{}
\begin{tikzpicture}

% Dummy
%\node at (0,6em) (dummy) {}; % blank space

% Unión
\node at (-3em,0em) (u0) {};
\node at ( 3em,0em) (u1) {};
\draw[line width=1pt, o-o] (u0) -- (u1) node [midway, sloped, anchor=center, above]{\footnotesize \textit{Unión 1}};

% Box 1
\node at (-15em,4em) (b0) {};
\node at (-15em,-4em) (b1) {};
\node at (-7em,-4em) (b2) {};
\node at (-7em,4em) (b3) {};
\draw[line width=1pt, -] (b0.center) -- (b1.center) -- (b2.center) -- (b3.center) -- (b0.center);
\node [text width=7em, align=center] at (-11em,0) (s1) {\footnotesize \textit{Subdominio 1} \scriptsize(condición de borde \textit{esencial})};

% Box 2
\node at (15em,4em) (b0) {};
\node at (15em,-4em) (b1) {};
\node at (7em,-4em) (b2) {};
\node at (7em,4em) (b3) {};
\draw[line width=1pt, -] (b0.center) -- (b1.center) -- (b2.center) -- (b3.center) -- (b0.center);
\node [text width=7em, align=center] at (11em,0) (s2) {\footnotesize \textit{Subdominio 2} \scriptsize(condición de borde \textit{natural})};

% Interfaz global 1
\node at (-7em,0em) (i10) {};
\node at (-5em,0em) (i11) {};
\draw[line width=1pt, -|] (i10.center) -- (i11.center) node [pos=1.0, text width=3em, above]{\footnotesize \textit{Interfaz local 1}}
node [pos=1.0, text width=3em, below]{\footnotesize \textit{Interfaz global 1}};

% Interfaz global 2
\node at (7em,0em) (i10) {};
\node at (5em,0em) (i11) {};
\draw[line width=1pt, -|] (i10.center) -- (i11.center) node [pos=1.0, text width=3em, above]{\footnotesize \textit{Interfaz local 1}}
node [pos=1.0, text width=3em, below]{\footnotesize \textit{Interfaz global 2}};

\end{tikzpicture}
\caption[Definición del problema de acoplamiento en el SSP de un reactor de investigación]
{Definición del problema de acoplamiento en el Segundo Sistema de Parada del reactor de investigación investigado en la sección \ref{3:mock-up}.}
\label{ra10-coupling}
\end{figure}

Las siguientes líneas conforman el archivo de configuración \texttt{mesh.cfg}:

\begin{Verbatim}[frame=single,commandchars=\\\{\}]
\textcolor{Gray}{! Nombre del problema}
\textcolor{Red}{*TITLE}
\textcolor{OliveGreen}{manifold mockup OPAL}

\textcolor{Gray}{! Número de subdominios}
\textcolor{Red}{*NUMBER_OF_COMPONENTS}
\textcolor{OliveGreen}{2     }\textcolor{Gray}{! Subdominio 3D y subdominio 0D}

\textcolor{Gray}{! Cantidad de interfaces en cada subdominio}
\textcolor{Red}{*NUMBER_OF_COMPONENT_INTERFACES}
\textcolor{OliveGreen}{1     }\textcolor{Gray}{! Subdominio (1): 1 interfaz}
\textcolor{OliveGreen}{1     }\textcolor{Gray}{! Subdominio (2): 1 interfaz}

\textcolor{Gray}{! Uniones en cada componente}
\textcolor{Red}{*CONECTIVITY_BETWEEN_COMPONENT_INTERFACES}
\textcolor{OliveGreen}{1 1   }\textcolor{Gray}{! Subdominio (1); Interfaz local (1): Unión 1}
\textcolor{OliveGreen}{2 1   }\textcolor{Gray}{! Subdominio (2); Interfaz local (1): Unión 1}

\textcolor{Gray}{! Pares de variables en cada componente}
\textcolor{Red}{*PAIR_VARIABLES_NUMBER_BY_COMPONENT_INTERFACE}
\textcolor{OliveGreen}{1     }\textcolor{Gray}{! Subdominio (1); interfaz local (1): 1 par}
\textcolor{OliveGreen}{1     }\textcolor{Gray}{! Subdominio (2); interfaz local (1): 1 par}

\textcolor{Gray}{! Componentes escalares en cada par de variables}
\textcolor{Red}{*PAIR_VARIABLES_NUMBER_OF_COMPONENTS}
\textcolor{OliveGreen}{1     }\textcolor{Gray}{! Subdominio (1); interfaz (1); par (1): 1 componente escalar}
\textcolor{OliveGreen}{1     }\textcolor{Gray}{! Subdominio (2); interfaz (1); par (1): 1 componente escalar}

\textcolor{Gray}{! Tipo de condición de borde en cada par}
\textcolor{Red}{*BOUNDARY_CONDITION_TYPE_BY_COMPONENT_PAIRS}
\textcolor{OliveGreen}{1     }\textcolor{Gray}{! Subdominio (1); interfaz (1); par (1): tipo Dirichlet (1)}
\textcolor{OliveGreen}{-1    }\textcolor{Gray}{! Subdominio (2); interfaz (1); par (1): tipo Neumann (-1)}

\textcolor{Gray}{! Cantidad de nodos en cada par de variables}
\textcolor{Red}{*PAIR_VARIABLES_NUMBER_OF_NODES}
\textcolor{OliveGreen}{1     }\textcolor{Gray}{! Subdominio (1); interfaz (1); par (1): 1 nodo}
\textcolor{OliveGreen}{1     }\textcolor{Gray}{! Subdominio (2); interfaz (1); par (1): 1 nodo}

\textcolor{Gray}{! Finalización}
\textcolor{Red}{*END}
\end{Verbatim}

Las siguientes líneas en el archivo de configuración \texttt{couple.cfg} definen la metodología numérica de resolución:

\begin{Verbatim}[frame=single,commandchars=\\\{\}]
\textcolor{Gray}{! Parámetro de evolución inicial}
\textcolor{Red}{*INITIAL_CONTINUATION_PARAMETER}
\textcolor{OliveGreen}{0.0}

\textcolor{Gray}{! Parámetro de evolución final}
\textcolor{Red}{*FINAL_CONTINUATION_PARAMETER}
\textcolor{OliveGreen}{15.0}

\textcolor{Gray}{! Cantidad de pasos de evolución}
\textcolor{Red}{*NUMBER_OF_CONTINUATION_STEPS}
\textcolor{OliveGreen}{1500}

\textcolor{Gray}{! Maxima cantidad de iteraciones del método de acoplamiento en cada}
\textcolor{Gray}{paso}
\textcolor{Red}{*MAXIMUM_NONLINEAR_ITERATIONS}
\textcolor{OliveGreen}{800}

\textcolor{Gray}{! Método de acoplamiento}
\textcolor{Red}{*NONLINEAR_SOLVER}
\textcolor{OliveGreen}{2} \textcolor{Gray}{! Broyden}

\textcolor{Gray}{! Tolerancia absoluta en residuos de variables acopladas}
\textcolor{Red}{*ABSOLUTE_TOLERANCE}
\textcolor{OliveGreen}{1.0e-15}

\textcolor{Gray}{! Tolerancia relativa en residuos de variables acopladas}
\textcolor{Red}{*TOLERANCE_NONLINEAR}
\textcolor{OliveGreen}{1.0e-08}

\textcolor{Gray}{! Iteraciones lineales máximas en la resolución de sistemas}
\textcolor{Gray}{algebraicos definidos en métodos implícitos}
\textcolor{Red}{*MAXIMUM_LINEAR_ITERATIONS}
\textcolor{OliveGreen}{800}

\textcolor{Gray}{! Pasos entre las reinicializaciones de la matriz jacobiana}
\textcolor{Red}{*JACOBIAN_INITIALIZATION}
\textcolor{OliveGreen}{100}

\textcolor{Gray}{! Parámetro de perturbación en cálculo de la matriz jacobiana mediante}
\textcolor{Gray}{diferencias finitas}
\textcolor{Red}{*JACOBIAN_PERTURBATION_PARAMETER}
\textcolor{OliveGreen}{0.1}

\textcolor{Gray}{! Máxima cantidad de evaluaciones de funciones en cada paso}
\textcolor{Red}{*MAXIMUM_FUNCTION_EVALUATIONS}
\textcolor{OliveGreen}{10000}

\textcolor{Gray}{! Órden de extrapolación de variables incógnitas}
\textcolor{Red}{*STATE_VARS_EXTRAPOLATION_ORDER}
\textcolor{OliveGreen}{1}

\textcolor{Gray}{! Órden de extrapolación de matriz jacobiana}
\textcolor{Red}{*JACOBIAN_EXTRAPOLATION_ORDER}
\textcolor{OliveGreen}{0}

\textcolor{Gray}{! Finalización}
\textcolor{Red}{*END}
\end{Verbatim}
