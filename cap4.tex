\chapter{Extensión al problema neutrónico-termohidráulico}
\label{chap4}
\chapterquote
{}
{alguien, alguna vez}

\section{Descripción del código Newton}
\label{4:newton}

con toda la experiencia ganada se desarrolló el código newton, gral
libre, open source, effiecietn error handling, etc.

features:
formas de conexión
mappers, 
alphas betas gammas y deltas,

Con la idea de extender estas capacidades al acoplamiento de programas cuyos códigos fuente no estuvieran disponibles,
así como a programas cuyos cálculos no dependieran exclusivamente de variables seteadas como condiciones de borde,
sino de otros parámetros generales del sistema,
se desarrolló un código más genérico de acoplamiento, el código \textbf{Newton} \cite{newton},
publicado en \textbf{Github} \cite{github} bajo Licencia Pública General (GPL por sus siglas en inglés, \textit{Public General License}) del Proyecto GNU colaborativo de software libre.
\textbf{Newton} es descripto en el Capítulo \ref{chap5}.


\section{Acople neutrónico-termohidráulico}
\label{4:neut-th}
breve descripción de los códigos utilizados

acople relap fermi

acople fermi cr

acople relap puma

esquema de variables intercambiadas y mapeos
