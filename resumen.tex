\begin{resumen}%
Los análisis de ingeniería actuales exigen estudios en sistemas cada vez más complejos. 
Éstos, a su vez, involucran subsistemas de características disímiles: principalmente diferentes tamaños y parámetros característicos. 
Por ejemplo, en los sistemas termohidráulicos es posible identificar distintos regímenes de flujo en tanques o en cañerías.
En ciertas ocasiones solo es de interés el detalle en algunos componentes,
necesitando modelar el resto del sistema para conservar la dinámica global.
En este trabajo se estudia una técnica que permite acoplar el modelado detallado de sistemas fluídicos bi- y tri- dimensionales 
con sistemas fluídicos más sencillos uni-dimensionales o cero-dimensionales. 
Cada subsistema se halla acoplado a los demás mediante los valores que toman las variables en las interfaces que comparten entre sí. 
El problema a resolver se reduce entonces a un sistema de ecuaciones cuyo tamaño depende de la cantidad de incógnitas en cada interfaz. 
Estas ecuaciones dependen, a su vez, de la física de cada subsistema y en general resultan ser no lineales.
Debido a esta característica, se investigan diferentes métodos de resolución iterativa.
Sobre el final del trabajo se extiende la técnica a acoples multifísicos
y se muestran algunos ejemplos de acoplamiento neutrónico-termohidráulico.
\end{resumen}

\begin{abstract}%
Current engineering analyzes require studies in increasingly complex systems.
These, in turn, involve subsystems of dissimilar characteristics: mainly different sizes and characteristic parameters.
For example, in thermohydraulic systems it is possible to identify different flow regimes in tanks or pipelines.
On some occasions it is only interesting to detail in some components, needing to model the rest of the system to preserve the global dynamics.
In this work we study a technique that allows the coupling of the detailed modeling of bi- and three-dimensional fluidic systems 
with simplified one-dimensional or zero-dimensional fluidic systems.
Each subsystem is coupled to the others by the values that the variables take on the interfaces they share with each other.
The problem to be solved is then reduced to a system of equations whose size depends on the number of unknowns in each interface.
These equations, in turn, depend on the physics of each subsystem and in general turn out to be non-linear.
Due to this characteristic, different iterative resolution methods are investigated.
On the end of the work the technique is extended to multiphysical couplings
and some examples of neutronic-thermohydraulic coupling are shown.
\end{abstract}


%%% Local Variables: 
%%% mode: latex
%%% TeX-master: "template"
%%% End: 
