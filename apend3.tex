\chapter{Acoplamiento de \textbf{Par-GPFEP}}
\label{C:pargpfep}
%\chapterquote{Negociemos Don Inodoro}{Fernando de la R\'{u}a, 2001}

\textbf{Par-GPFEP} (programa de elementos finitos de propósito general paralelizado, el nombre se debe a sus siglas
en inglés, \textit{Parallel General Purpose Finite Element Program})
es un código escrito en \texttt{Fortran 90} y en \texttt{C},
desarrollado inicialmente por G. Buscaglia, S. Felicelli, E. Dari, A. Lew y M. Raschi,
y que ha ido teniendo contribuciones a partir de diferentes trabajos y tesis realizadas
en el Departamento de Mecánica Computacional de la Comisión Nacional de Energía Atómica (\cite{gpfep}, \cite{pargpfep}).

En el presente trabajo se utilizaron y modificaron las rutinas relacionadas con el bucle de resolución general
y el ensamblaje de matrices para las ecuaciones de \textit{Navier-Stokes}, de modelos de turbulencia, flujo multifase y transporte de energía.
Además, se implementaron las rutinas necesarias para realizar el acoplamiento mediante el modelo de comunicación por paso de mensajes
a partir de la conexión a puertos publicados por el código \textit{maestro}, de modo que pudiera comunicarse con \textbf{Coupling} y con \textbf{Newton} de forma exitosa.
Éstos trabajos son comentados en la sección \ref{pargpfep-impl}.
También se desarrolló un programa para la identificación de condiciones de borde sobre conjuntos de nodos específicos, el cual es comentado en la sección \ref{genbco}.

\section{Implementación de la arquitectura de acoplamiento en \textbf{Par-GPFEP}}
\label{pargpfep-impl}

La estructura principal de acoplamiento se compone de cuatro instancias genéricas.
Estas cuatro instancias fueron descriptas en la sección \ref{2:arquitectura-mpi} y se programaron en la rutina \texttt{pargpfep\_control.for}.
Cada una de estas intancias es llamada desde el bucle de resolución general en momentos adecuados.
Pero para que este acoplamiento resultara exitoso, fue necesario además implementar una serie de funciones extra:
\begin{itemize}
\item Análisis de palabras clave del archivo de configuración \texttt{gpfep.cfg} específicas a la formulación del problema de acoplamiento (implementado en el archivo \texttt{cfgreadingCoup.for}).
\item Seteo paralelizado de variables sobre grupos de superficie conforme a perfiles previstos:
  \begin{itemize} 
  \item para condiciones de borde \textit{esenciales} (modificaciones en el archivo \texttt{bcfemco.for});
  \item para condiciones de borde \textit{naturales} (modificaciones en el archivo \texttt{nssurf.for});
  \end{itemize}
\item Cálculo integral paralelizado de variables sobre grupos de superficie (implementado en el archivo \texttt{sendValuesCoupling.for}).
\item Cálculo de campos vectoriales gradientes a partir de campos escalares (modificaciones en el archivo \texttt{gpmain.F}).
\item Condicionamiento sobre el avance evolutivo para respetar las órdenes de reinicio o continuación recibidas desde programa \textit{maestro} (modificaciones en el archivo \texttt{gpmain.F}).
\item Resguardo (\textit{back-up}) de las soluciones calculadas debido a la posibilidad de reinicio en determinados pasos de cálculo (modificaciones en el archivo \texttt{gpmain.F}).
\end{itemize}

\section{El programa \textbf{genbco}}
\label{genbco}

Normalmente, las condiciones de borde de tipo \textit{Dirichlet} se establecen con los programas \textbf{genbcc} y \textbf{gpboco}.
Estos programas imponen para alguna variable dada, el mismo valor a todos los nodos que pertenecen a un grupo de superficie seleccionado.
Durante el trabajo realizado, resultó necesario diferenciar los nodos sobre los que se aplicaba las condiciones de borde.
Por ejemplo, era necesario sustraer los nodos de las aristas a los nodos de una cara, englobar varias aristas en un mismo grupo, englobar varias caras en un mismo grupo, etc.
Con estos fines fue desarrollado el código \textbf{genbco}.
\textbf{genbco} unifica las funciones previamente desempeñadas por los códigos \textbf{genbcc} y \textbf{gpboco}.
Los archivos de entrada necesarios para su uso son el archivo de nodos de superficie y conectividades \texttt{.sur} específico de \textbf{Par-GPFEP}
y el archivo de configuración \texttt{genbco.cfg}.
En \texttt{genbco.cfg} se especifican los conjuntos de nodos sobre los que se desea imponer condiciones de borde de tipo \textit{Dirichlet} para cada variable de cálculo.
Esta especificación se realiza a partir de operaciones \textit{booleanas} sobre los grupos originales de superficie.
Las operaciones permitidas son:
\begin{itemize}
\item unión (U);
\item intersección (I);
\item sustracción (m).
\end{itemize}
Sobre cada uno de los conjuntos se especifican, además, los valores iniciales de contorno.
El archivo de salida \texttt{gp.bco} contiene una lista con todos los nodos y sus condiciones de borde para cada variable y está listo para ser leído posteriormente por \textbf{Par-GPFEP}.


\section{Ejemplo de uso de \textbf{Par-GPFEP} en forma acoplada}
\label{pargpfep-coup}

A continuación se comentan las consideraciones necesarias en \textbf{Par-GPFEP}
para la resolución del problema \textit{Análisis del segundo sistema de parada de un reactor de investigación} analizado en \ref{3:mock-up}.

En primer lugar, es necesario notificar en el archivo de configuración \texttt{gpfep.cfg} que las condiciones de borde de algunas interfaces están acopladas.
El llamado a las funciones de acoplamiento se activa con la siguiente bandera:
\begin{Verbatim}[frame=single,commandchars=\\\{\}]
\textcolor{Gray}{! Switch para acople}
\textcolor{Red}{*COUPLING_SWITCH}
\textcolor{OliveGreen}{1}
\end{Verbatim}

El método de comunicación utilizado es el de conexión a puertos previamente publicados por el programa \textit{maestro}.
En este método es necesario definir un código de identificación para cada programa \textit{esclavo}:
\begin{Verbatim}[frame=single,commandchars=\\\{\}]
\textcolor{Gray}{! ID del código}
\textcolor{Red}{*CODE_ID}
\textcolor{OliveGreen}{1}
\end{Verbatim}

Las siguientes tarjetas se utilizan para definir las interfaces y las variables de acople:
\begin{Verbatim}[frame=single,commandchars=\\\{\}]
\textcolor{Gray}{! Cantidad de interfaces de acople}
\textcolor{Red}{*INTERFACES}
\textcolor{OliveGreen}{1}

\textcolor{Gray}{! Área [m2] de cada interfaz}
\textcolor{Red}{*INTERFACES_AREAS}
\textcolor{OliveGreen}{0.05067}

\textcolor{Gray}{! Grupo de la malla correspondiente a cada interfaz}
\textcolor{Red}{*INTERFACES_GROUPS}
\textcolor{OliveGreen}{3}

\textcolor{Gray}{! Pares de variables en cada interfaz}
\textcolor{Red}{*UNKNOWN_PAIRS_PER_INTERFACE}
\textcolor{OliveGreen}{1}

\textcolor{Gray}{! Tipo de par intercambiado en cada interfaz:}
\textcolor{Gray}{! (Tipo 1: caudal-presión; Tipo 2: temperatura-flujo de calor)}
\textcolor{Red}{*INTERFACES_PAIRS_TYPES}
\textcolor{OliveGreen}{1}
\end{Verbatim}

A continuación es necesario definir el tipo de perfil que va a ser utilizado para calcular los vectores de velocidades en cada interfaz con condición de borde de tipo \textit{Dirichlet}.
\begin{Verbatim}[frame=single,commandchars=\\\{\}]
\textcolor{Gray}{! Tipo de perfil de velocidades para la componente x, componente y,}
\textcolor{Gray}{! componente z, en cada interfaz con seteo de caudal}
\textcolor{Gray}{! (Tipo 0: nulo; Tipo 1: uniforme; Tipo 2: parabólico 2D; Tipo 3:}
\textcolor{Gray}{! parabólico 3D )}
\textcolor{Red}{*FLOW_PROFILES}
\textcolor{OliveGreen}{1} \textcolor{Gray}{! Cantidad de perfiles}
\textcolor{OliveGreen}{1 0 0} \textcolor{Gray}{! Tipo para cada componente escalar (x,y,z)}
\end{Verbatim}

Si bien en el modelado del problema se utiliza un perfil de velocidades plano, en ciertas ocasiones puede requerirse el uso de perfiles parabólicos.
Para calcular estos perfiles es necesario proveer algunas coordenadas de referencia.
Si se hubiera utilizado un perfil parabólico tri-dimensional,
y dado que la sección es circular,
debe proveerse la posición del centro de la misma y el valor de su radio del siguiente modo:
\begin{Verbatim}[frame=single,commandchars=\\\{\}]
\textcolor{Gray}{! Coordenadas del punto de referencia de cada interfaz con seteo de}
\textcolor{Gray}{! caudal}
\textcolor{Red}{*INTERFACES_REFERENCES}
\textcolor{OliveGreen}{0.0 0.0 0.0}

\textcolor{Gray}{! Radio en cada interfaz con seteo de caudal}
\textcolor{Red}{*INTERFACES_RADIUS}
\textcolor{OliveGreen}{0.127}
\end{Verbatim}

Previo al modelado de futuros problemas acoplados, debe verificarse la implementación de ciertas particularidades
bajo la palabra clave \texttt{USUARIO} en los archivos comentados en la sección \ref{pargpfep-impl}.
Por ejemplo, cuando la variable de acople en una interfaz dada es una fuerza, ésta puede recibirse con diferentes signos según la formulación del problema,
por lo que es responsabilidad del usuario su correcto uso.
