\chapter{Estrategia de acoplamiento}
\label{chap2}
\chapterquote
%~ {We build too many walls and not enough bridges.}
{Construimos demasiadas murallas y no suficientes puentes.}
{Isaac Newton, 1643-1727}

\section{Paradigma maestro-esclavo}
\label{2:maestro-esclavo}

El modelo de comunicación utilizado en el trabajo recibe el nombre de \textit{maestro-esclavo} \cite{maestro-esclavo}.
En este modelo, existe un programa \textit{maestro} que tiene el control unidireccional sobre los demás programas, que actúan bajo el rol de \textit{esclavos}.
Cada código \textit{esclavo} se encarga de calcular el valor de las incógnitas en las interfaces de acople mediante las ecuaciones \ref{ecuaciones-modelos}.
El código \textit{maestro} recibe estos valores y con ellos resuelve las ecuaciones de residuos \ref{ecuaciones-residuos}.
En base a estos residuos propone\footnote{
Esta propuesta reside en el método de resolución de ecuaciones no lineales seleccionado, ver sección \ref{1:metodos}.
} nuevos valores para las inćognitas en las interfaces de acople y se los envía a sus \textit{esclavos}.
Así también, es función del código \textit{maestro} enviarles órdenes de comenzar el cálculo en un dato paso de evolución, reiniciar el cálculo o incluso abortar.

Cabe notar que cada código \textit{esclavo} podría ejecutarse en varios procesos, paralelizando sus cálculos,
ya sea mediante memoria compartida como mediante memoria distribuida.
Incluso podrían lanzarse diversos procesos del código \textit{maestro} en la resolución de algún problema.
Debido a la complejidad en destinatarios de mensajes, cantidades y tipos de variables a compartir,
es necesario definir una estrategia clara de comunicación
que permita acoplar diversos códigos de manera genérica, segura y eficaz.
La estrategia definida es comentada en la sección \ref{2:estructura}.

\section{Códigos maestros utilizados}
\label{2:maestros}

En este trabajo se utilizaron dos códigos maestros para la resolución de los problemas.
El primer código \textit{maestro} utilizado es el código \textbf{Coupling} desarrollado por XXX en XXX.
\textbf{Coupling} permite acoplar códigos mediante funciones del estándard Interfaz de Paso de Mensajes (MPI por sus siglas en inglés, \textit{Message Passing Interface}).
Esta estrategia requiere el código fuente de los programas \textit{esclavos} para implementar las funciones adecuadas.
El código maestro está diseñado de tal forma que los códigos acoplados resuelvan ecuaciones del tipo \ref{ecuaciones-modelos}
para incógnitas en interfaces con condiciones de borde de tipo \textit{Dirichlet} o de tipo \textit{Neumann}.

Con la idea de extender estas capacidades al acoplamiento de programas cuyos códigos fuente no estuvieran disponibles,
así como a programas cuyos cálculos no dependieran exclusivamente de variables seteadas como condiciones de borde,
sino de otros parámetros generales del sistema,
se desarrolló un código más genérico de acoplamiento, el código \textbf{Newton} \cite{newton},
que será descripto en el Capítulo \ref{chap5}.

\section{Modelos de comunicación}
\label{2:comunicacion}

La configuración \textit{maestro-esclavo} requiere la ejecución de múltiples programas independientes.
Al mismo tiempo, cada código podría estar corriendo en forma paralelizada\footnote{
En general, cada programa \textit{esclavo} es un programa que ya ha sido utilizado y validado para algún tipo de cálculo.
Si el código está paralelizado, en base a estudios de \textit{speedup} se podría tener cierta experiencia en su modo de ejecución óptimo para alguna tarea dada.
Esta ejecución podría requerir múltiples nodos en un clúster, por ejemplo.
}.
Debido a esta complejidad es necesario planificar la estrategia de comunicación considerando la distribución del cálculo en múltiples procesadores,
con el objetivo de no perder generalidad en la herramienta desarrollada.
Los modos de comunicación implementados son los siguientes:
\begin{itemize}
\item paso de mensajes: este modo es implementado para comunicar procesos de programas en los cuales es posible modificar los códigos fuente;
\item lectura y escritura de archivos de entrada y salida: este modo es implementado para comunicar procesos de programas que serán tratados como cajas negras.
\end{itemize}

Si bien este tipo de comunicaciones remotas son mucho más lentas que las
comunicaciones locales, en general su latencia es despreciable frente al tiempo de cálculo de los procesos \textit{esclavos}.

\subsection*{Paso de mensajes}
\label{2:mpi}

En sistemas de memoria distribuida, el paso de mensajes es un método de
programación utilizado para realizar el intercambio de datos entre los
procesadores \cite{comunicacion}. En estos sistemas, los datos son enviados de un procesador a otro
utilizando mensajes. El paso de mensajes involucra la transferencia de datos desde un proceso
que envía a un proceso que recibe. El proceso que envía,
necesita conocer la localización, el tamaño y el tipo de los datos, así como el
proceso destino.

Estas funcionalidades se implementaron siguiendo el protocolo MPI. 
MPI es una especificación de paso de mensajes aceptada como estándar
por todos los fabricantes de computadores. 
El objetivo principal de MPI es proporcionar un estándar para escribir
programas (lenguajes \textit{C}, \textit{C++}, \textit{Fortran}) con paso de mensajes. De esta forma, se
pretende mejorar la portabilidad, el rendimiento, la funcionalidad y
la disponibilidad de las aplicaciones.

Se utilizaron dos implementaciones alternativas.
En la primera, cada código \textit{esclavo}, así como el código \textit{maestro},
son ejecutados de manera independiente en uno (SISD por sus siglas en inglés, Single Instruction, Single Data) o múltiples procesos (SIMD por sus siglas en inglés, Single Instruction, Multiple Data).
El código \textit{maestro} publica una serie de puertos a los cuales cada código \textit{esclavo} puede conectarse\footnote{
Para que los programas puedan encontrar los puertos publicados, es necesario que todos ellos pertenezcan a un mismo servicio de comunicación generado por el \textit{ompi-server}.
}.
Una vez aceptadas las conexiones, los programas pueden intercambiar mensajes siguiendo una lógica preestablecida.
Cuando ya no es necesario que dos programas continúen comunicándose, se cierran las conexiones.

En la otra implementación, todos los programas son ejecutados al mismo tiempo (MIMD por sus siglas en inglés, Multiple Instruction, Multiple Data).
En este tipo de ejecuciones todos los procesos cuentan con un único comunicador original, \textit{MPI\_COMM\_WORLD}, y por ello es necesario crear nuevos grupos de procesos y de comunicadores.
Una vez establecidos los comunicadores, los programas pueden intercambiar mensajes siguiendo la misma lógica preestablecida en el modelo previo.
Si bien esta implementación no permite la posibilidad de realizar nuevas conexiones una vez que los programas han sido ejecutados,
son mucho más seguras, ya que no dependen del éxito de encontrar los puertos requeridos para las conexiones.

\subsection*{Lectura y escritura de archivos de entrada y salida}
\label{2:io}

La comunicación mediante lectura y escritura de archivos se implementó para demostrar la capacidad de acoplar códigos cuyos códigos fuente no son capaces de ser modificados.
La idea principal es ejecutar corridas simples del código \textit{esclavo} administradas desde el código \textit{maestro}.
Para ello el código \textit{maestro} escribe en el archivo de entrada del programa \textit{esclavo} todos los parámetros necesarios para la ejecución del cálculo,
ordena su ejecución, y espera a que este termine y luego realiza una búsqueda de los valores de las variables de interés en archivos de salida.
En problemas de evolución, el código \textit{maestro} debe notificar en el archivo de entrada el parámetro de evolución,
así como otros valores de variables de estado del paso previo.

La ejecución de programas \textit{esclavos} se implementó de dos formas alternativas.
La primera forma es mediante el uso de la función \textit{system} de la librería de \textit{c}.
Esta función deja al proceso que la ejecuta en pausa hasta que el programa \textit{esclavo} finaliza, cuando ella retorna algún mensaje de error o de éxito.
La ventaja de esto es que no debe implementarse alguna función extra para conocer cuándo leer los archivos de salida del programa \textit{esclavo}.
Sin embargo, no es posible disparar múltiples procesos de un programa mediante la función \textit{system} desde un proceso que actualemente utiliza \textit{MPI}.
Ésta prohibición es necesaria para controlar el disparo de procesos.
Para este tipo de ejecuciones, existe la función \textit{MPI\_Comm\_Spawn} de \textit{MPI}, que se implementó como forma alternativa de ejecución de programas \textit{esclavos}.
Esta función permite especificar la cantidad de procesos de ejecución del programa a disparar.
El problema es que la función devuelve el control al programa \textit{maestro} de forma instantánea, sin esperar a que el código disparado finalice,
por lo que, en principio, debe implementarse alguna función extra para saber cuándo es posible leer el archivo de salida.

Es necesario notar que este modelo de comunicación no es tan eficiente como el de intercambio de mensajes,
ya que la lectura y escritura de archivos consume mayores recursos de tiempo, por lo que, siempre que fuera posible, es recomendable implementar el otro modelo.
Además, requiere la programación de rutinas extras específicas dedicadas a la escritura de archivos de entrada y lectura de archivos de salida de distintos códigos \textit{esclavos}.

\subsection*{Estructura de comunicación implementada}
\label{2:estructura}

Se desarrollaron funciones híbridas para códigos maestros con la finalidad de cubrir todas las formas de comunicación descriptas en la sección previa.
En el caso de comunicación por intercambio de mensajes, 
la estrategia definida establece comunicaciones siempre entre un único proceso del código \textit{maestro}, el proceso \textit{raíz},
y un único proceso de cada código \textit{esclavo} (sus propios procesos \textit{raíces}\footnote{
Es responsabilidad del código \textit{esclavo} la comunicación de los datos recibidos por el proceso \textit{raíz} a los demás procesos.
}).
En el caso de lectura y escritura de archivos y ejecución de programas \textit{esclavos},
la estrategia definida paraleliza las responsabilidades entre todos los procesos lanzados del código \textit{maestro}.
El esquema \ref{esquema-comunicacion} resume la estrategia de comunicación.


\begin{figure}
\centering{}
\begin{tikzpicture}[
	block1/.style={
	draw,
	fill=white,
	rectangle, 
	minimum width={width("Proceso $N_m$")+2pt},
	minimum height={15pt},
	font=\small},
	block2/.style={
	draw,
	fill=white,
	rectangle,
  fill=blue!20,
  text centered, 
  rounded corners,
	minimum width={width("Esclavo N")+2pt},
	minimum height={15pt},
	font=\large},
	connector/.style={
	-o,
	line width=0.3mm
	},
  arrow1/.style={
  -{Latex[length=3mm, width=1mm]}
  },
  arrow2/.style={
  {Latex[length=3mm, width=1mm]}-{Latex[length=3mm, width=1mm]}
  }]

	% Dummy
	\node at (0,10em)             (dummy) {}; % blank space

	% Master
	\node [block2] at (0,8em)             (master) {\textbf{Maestro}};
	\node [block1, below=1em of master]   (masterp1) {...};
	\node [block1, left=1em of masterp1]  (masterp0) {Proceso $0_m$};	
	\node [block1, right=1em of masterp1] (masterpN) {Proceso $N_m$};

	% Slave 1
	\node [block2] at (-6.5,0em)    		(slave1) {\textbf{Esclavo 1}};
	\node [block1, below=1em of slave1] (s1p0) 	 {Proceso $0_1$};
	\node [block1, below=1em of s1p0]   (s1p1) 	 {...};
	\node [block1, below=1em of s1p1] 	(s1pN) 	 {Proceso $N_1$};
	
	% Slave 2
	\node [block2] at (-3.5,-4em)    		(slave2) {\textbf{Esclavo 2}};
	\node [block1, below=1em of slave2] (s2p0) 	 {Proceso $0_2$};
	\node [block1, below=1em of s2p0]   (s2p1) 	 {...};
	\node [block1, below=1em of s2p1] 	(s2pN) 	 {Proceso $N_2$};
	
	% Slave 3
	\node [block2] at (0,0em)      		  (slave3) {\textbf{Esclavo 3}};
	\node [block1, below=1em of slave3] (s3p0) 	 {Proceso $0_3$};
	\node [block1, below=1em of s3p0]   (s3p1) 	 {...};
	\node [block1, below=1em of s3p1] 	(s3pN) 	 {Proceso $N_3$};
	
	% Slave 4
	\node [block2] at (3.5,-4em)   		  (slave4) {\textbf{Esclavo $i$}};
	\node [block1, below=1em of slave4] (s4p0) 	 {Proceso $0_i$};
	\node [block1, below=1em of s4p0]   (s4p1) 	 {...};
	\node [block1, below=1em of s4p1] 	(s4pN) 	 {Proceso $N_i$};
	
	% Slave N
	\node [block2] at (6.5,0em)    		  (slave5) {\textbf{Esclavo N}};
	\node [block1, below=1em of slave5] (s5p0) 	 {Proceso $0_N$};

	% Processes
	\draw[connector] (master.south) -- ($(masterp0.north)+(0,0.2em)$);
	\draw[connector] (master.south) -- ($(masterp1.north)+(0,0em)$);
	\draw[connector] (master.south) -- ($(masterpN.north)+(0,0.2em)$);

	\draw[connector] (slave1.west) -- ++(-1em,0) |- (s1p0.west);
	\draw[connector] (slave1.west) -- ++(-1em,0) |- (s1p1.west);
	\draw[connector] (slave1.west) -- ++(-1em,0) |- (s1pN.west);

	\draw[connector] (slave2.west) -- ++(-1em,0) |- (s2p0.west);
	\draw[connector] (slave2.west) -- ++(-1em,0) |- (s2p1.west);
	\draw[connector] (slave2.west) -- ++(-1em,0) |- (s2pN.west);

	\draw[connector] (slave3.west) -- ++(-1em,0) |- (s3p0.west);
	\draw[connector] (slave3.west) -- ++(-1em,0) |- (s3p1.west);
	\draw[connector] (slave3.west) -- ++(-1em,0) |- (s3pN.west);

	\draw[connector] (slave4.west) -- ++(-1em,0) |- (s4p0.west);
	\draw[connector] (slave4.west) -- ++(-1em,0) |- (s4p1.west);
	\draw[connector] (slave4.west) -- ++(-1em,0) |- (s4pN.west);

	\draw[connector] (slave5.west) -- ++(-1em,0) |- (s5p0.west);


	% Comunicators
	\draw[arrow2] ($(masterp0.south)-(1em,0)$) to[out=-90, in=15, distance=2em] node [midway, sloped, anchor=center, above]{\textit{MPI}} (s1p0.east);

	\draw[arrow2] ($(masterp0.south)+(0em,0)$) to[out=-90, in=35, distance=3em] node [midway, rotate=90, anchor=center, above]{\textit{MPI}} (s2p0.east);

	\draw[arrow1] ($(masterp0.south)+(1em,0)$) to[out=-90, in=105, distance=2em] node [midway, sloped, anchor=center, above]{\textit{Ejecución}} (slave3.north);

	\draw[arrow1] (masterp1.south) to[out=-90, in=105, distance=2em] node [midway, sloped, anchor=center, above]{\textit{Ejecución}} (slave4.north);

	\draw[arrow1] (masterpN.south) to[out=-90, in=105, distance=2em] node [midway, sloped, anchor=center, above]{\textit{Ejecución}} (slave5.north);

\end{tikzpicture}
\caption[Esquema de comunicación implementado]{Esquema de comunicación entre los programas \textit{esclavos} y el programa \textit{maestro} implementado en el acoplamiento de códigos.
Los \textit{esclavos} se comunican solo con el \textit{maestro} y no intercambian datos entre sí.
Se utilizan dos modelos de comunicación diferentes.
En el primero, cada código \textit{esclavo} es comunicado con el código \textit{maestro} a través de intercambio de mensajes por \textit{MPI}.
Como podrían correr en modo serial o paralelo, solo sus procesos \textit{raíces} establecen la comunicación con el proceso \textit{raíz} del programa \textit{maestro}.
En el segundo modelo de comunicación, los códigos \textit{esclavos} son directamente \textit{ejecutados} por el programa \textit{maestro} en uno o varios procesos.
En este modelo, la comunicación se establece solo mediante lectura y escritura de archivos.}
\label{esquema-comunicacion}
\end{figure}

\section{Arquitectura de acoplamiento montada en códigos \textit{esclavos} comunicados por paso de mensajes}
\label{2:arquitectura-mpi}

En general, los códigos \textit{esclavos} son programas de cálculo particulares que no han sido diseñados para mantenerse acoplados a otros códigos.
En esta sección se demuestra cómo mediante unas mínimas modificaciones en sus rutinas es posible implementar un acoplamiento eficiente.
Las acciones de acoplamiento deben ser llamadas en 4 instancias diferentes:
\begin{enumerate}
\item al principio del programa;
\item al principio de cada paso de evolución;
\item al finalizar cada paso de evolución;
\item al finalizar el programa.
\end{enumerate}
Los problemas que no involucran evolución de variables pueden ser tratados como problemas con un solo paso de evolución.
El programador podría definir una nueva variable \textit{booleana} que a modo de bandera indique cuándo se está realizando un cálculo acoplado para ingresar o no en las instancias nombradas.
A continuación se describen las instancias de acplamiento.

\subsection*{Acoplamiento en instancia 1: al principio del programa}

En esta instancia es necesario establecer la comunicación \textit{MPI} entre el proceso \textit{raíz} del código \textit{esclavo} y el código maestro.
Si ambos programas han sido ejecutados en el esquema \textit{MIMD} los pasos a realizar son los siguientes:
\begin{itemize}
\item creación de grupo global de procesos;
\item creación de subgrupo local de procesos;
\item creación de un comunicador dentro del subgrupo previo, necesario para el paso de mensajes dentro del programa;
\item creación de un grupo entre el proceso \textit{raíz} del programa esclavo y el proceso \textit{raíz} del programa \textit{maestro};
\item creación de un comunicador en el grupo previo, necesario para el paso de mensajes de acople.
\end{itemize}
Si en cambio, el programa ha sido ejecutado en forma independiente, los pasos a realizar son los siguientes:
\begin{itemize}
\item búsqueda del puerto publicado por el el proceso \textit{raíz} del programa \textit{maestro};
\item conexión del proceso \textit{raíz} del programa esclavo a este puerto y creación del comunicador.
\end{itemize}
Una vez implementada la comunicación, el código \textit{esclavo} puede recibir datos generales 
(como parámetros de evolución iniciales, cantidad de pasos de evolución, cantidad de incógnitas en interfaces de acople, cantidad de interfaces de acople, etc.),
y chequear la consistencia con los datos propios del programa.
Si es necesario, los datos locales pueden ser cambiados notificando al usuario.

\subsection*{Acoplamiento en instancia 2: al principio de cada paso de evolución}

La estrategia de acoplamiento se define entre el parámetro de evolución inicial $t_{coup,0}$ y el parámetro final $t_{coup,N}$,
con $N+1$ pasos de acoplamiento cada $\Delta t_{coup}=\frac{t_{coup,N} - t_{coup,0}}{N}$.
Si bien el código \textit{esclavo} debe intercambiar mensajes en cada uno de estos pasos, es posible que además utilice subpasos de evolución $\Delta t_{local}$ locales\footnote{
Distintos programas pueden tener diferentes requisitos sobre el parámetro de evolución, dependiendo de la física que resuelven.
Algunos, por ejemplo, podrían estar resolviendo transitorios fluidodinámicos, en los que es de interés mantener por debajo de algún valor 
ciertos parámetros (como el número de \textit{Courant}) dependientes del paso de tiempo.
Otros, en cambio, pueden no tener este requisito.
La idea es que cada programa \textit{esclavo} devuelva al programa \textit{maestro} el mejor resultado posible al cabo de $\Delta t_{coup}$.
}. En estos casos, se implementa una estrategia de interpolación de los valores de las variables en las interfaces de acoplamiento entre los pasos de evolución acoplados.

Al principio de cada paso acoplado de cálculo el código \textit{esclavo} recibe valores supuestos para las incógnitas que se toman como dato en las interfaces de acople,
en función de la estretegia implementada (ver sección \ref{1:abordaje}).
El programa resuelve el paso $\Delta t_{coup}$ en base a ellos.

\subsection*{Acoplamiento en instancia 3: al finalizar cada paso de evolución}

Una vez resuelto cada $\Delta t_{coup}$, el programa envía al programa \textit{maestro} los valores de las variables que se han definido como incógnitas en las interfaces de acoplamiento.
Tras este envío, el programa queda en espera de órden para continuar.
Mientras, el programa \textit{maestro} recepciona los valores de las incógnitas calculados por los demás códigos \textit{esclavos}.
Con estos valores resuelve las ecuaciones de residuos.
Si el módulo del residuo cae por debajo de cierta tolerancia prefijada el código \textit{maestro} acepta los resultados y envía a sus \textit{esclavos} la órden de continuar con el cálculo.
En caso contrario, puede enviarles la órden de volver a calcular el mismo paso de acoplamiento, o incluso de abortar el cálculo.

\subsection*{Acoplamiento en instancia 4: al finalizar el programa}

Antes de finalizar el programa, es necesario cerrar las conexiones, liberar los grupos y los comunicadores establecidos.
