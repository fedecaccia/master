\chapter{Técnicas de acoplamiento}
\label{chap2}
\chapterquote
{}
{alguien, alguna vez}

\section{Paradigma maestro-esclavo}
\label{2:maestro-esclavo}

\section{Códigos maestros utilizados}
\label{2:maestros}
Dos códigos:
Código coupling desarrollado previamente x ...., para acoplar pares D2N.

Visto que existen códigos cerrados incapaces de ser comunicados por MPI, 
o que exigen otra formulación de residuos,
ydurante la tesis se realizó un código más general, Newton, descripto en último capítulo.
Con múltiples tipos de conexión (mpi-port, mpi-comm, I/O spawn and system),
mappers, etc.

\section{Formas de comunicación implementadas}
\label{2:comunicación}

mpi vs compartida, ver documentos/biblio-maestria/mpi

técnicas mpi-port, mpi-comm, I/O spawn and system
implementación: 

grupos, 
puertos, 

spawn,
barrier after spawn

system

\section{Arquitectura de acoplamiento montada en códigos esclavos}
\label{2:arquitectura}
